%%cover
\input epsf
\usepackage{graphicx}
\usepackage{pstricks}
\usepackage{amsmath}

% now defined in dalibo.sty
%\newrgbcolor{darkgreen}        {0.23 0.65 0.20} %3ca832
%\definecolor{darkgreen}{rgb}   {0.23,0.65,0.20}
%\definecolor{stronggreen}{rgb} {0.29,0.41,0.07} % 4b6912

\begin{document}
\thispagestyle{empty}

\noindent
\begin{pspicture}(0,2)(17,22)
  %Use psgrid while placing your graphics element
  %\psgrid

  % place the big logo on cover page
  \rput[b](8.5,19){\epsfxsize=182mm \epsfbox{$doc.logo}}

  % place a background spiral on the page
  \rput(8.5,10.2){\epsfbox{$doc.bg}}

  %\psline[linewidth=4mm,linecolor=darkgreen](0,12.3)(17,12.3)
  %\psline[linewidth=4mm,linecolor=darkgreen](0,10)(17,10)

  \rput[rB](16.5,12.5){\huge $doc.type.encode('latin-1')}
  \rput[rB](16.5,10.2){\Huge \textbf{$doc.title.encode('latin-1')}}

\end{pspicture}


% paramétrage des listes à puces
% comme on utilise le package french, il doit être fait après le 
% \begin{document}

% exemple avec fichier eps pour puce graphique
%\renewcommand\labelitemi{\epsfxsize=12pt \epsfbox{/tmp/spirale.eps}}

\renewcommand\labelitemi{\textcolor{darkgreen}{$\bullet$}}
%\renewcommand\labelitemii{\textcolor{darkgreen}{$\ast$}}
\renewcommand\labelitemii{\textcolor{stronggreen}{$\Rightarrow$}}
\renewcommand\labelitemiii{\textcolor{stronggreen}{$\diamond$}}
\renewcommand\labelitemiv{\textcolor{stronggreen}{$\triangleright$}}

%%dalidoc
\usepackage{eurosym} 
\usepackage[latin9]{inputenc} 
\let ¤ = \euro 

\usepackage{fancyhdr}
\pagestyle{fancy}

%% Intégration style Dom
%% 
%% JPAREM 2006023 : je veux bien mais IDX* pas glop
\usepackage{fancybox}
\usepackage{geometry}
\usepackage{float}
\usepackage{here}
\usepackage{color}
\usepackage[grey,newcentury]{quotchap}
\usepackage{longtable}
\usepackage{multirow}
\makeatletter
\let\IDXbacksl\@backslashchar    % urk 
\let\IDX@oldtt\texttt            % urk
\def\texttt#1{{%
\@ifundefined{NoAutoSpaceBeforeFDP}{}{\NoAutoSpaceBeforeFDP}%
\IDX@oldtt{#1}}}
\makeatother
\DeclareGraphicsExtensions{.epsf,.png,.jpg}

\makeatletter
\renewcommand*{\chapnumfont}{%
    \usefont{T1}{\@defaultcnfont}{b}{n}\fontsize{70}{90}\selectfont%
    \color{chaptergrey}}
\makeatother

\cornersize*{6pt} % For fancybox

\floatstyle{ruled}
\widowpenalty=10000
\clubpenalty=10000

%% Pied de page dalibo
\lhead{}
\chead{}
\rhead{$doc.title.encode('latin-1')}
\lfoot{
 \begin{tabular}{p{30mm}}
 \includegraphics[width=30mm]{$doc.slogo}
 \end{tabular}
}
\cfoot{\footnotesize{dalibo S.A.R.L au capital de 8192 \euro \\
190, Avenue du Général Leclerc\\
78220 VIROFLAY\\
http://www.dalibo.com/}
}
\rfoot{\thepage\ / \pageref*{LastPage}}

%% On veut le pied de page même sur les pages de chapter
%% et avec le trait de 0.4 point et tout, hein
\pagestyle{fancyplain}
\fancypagestyle{plain}{%
  \renewcommand\footrulewidth{0.4pt}
}

\renewcommand{\headrulewidth}{0.4pt}
\renewcommand{\footrulewidth}{0.4pt}

\usepackage{helvet}

\renewcommand{\familydefault}{phv}

%Parametrage pour une feuille A4 pleine (merci SBI)
\evensidemargin = 30mm
\oddsidemargin = 30mm
\voffset=-1in
\topmargin = 17mm
\headheight = 14.5mm
\headsep = 15mm
\hoffset=-1in
\marginparsep = 0pt
\marginparwidth = 0pt
\footskip = 20mm
\textwidth=162mm
\textheight=200mm
\paperwidth=210mm
\paperheight=297mm
\parindent=0pt
\parskip=5pt
%fin parametrage A4 plein

\usepackage{lastpage}
\usepackage{indentfirst}

\definecolor{darkgreen}{rgb}   {0.23,0.65,0.20}
\definecolor{stronggreen}{rgb} {0.29,0.41,0.07} % 4b6912
\hypersetup{linkcolor=darkgreen,urlcolor=stronggreen,colorlinks=true}
